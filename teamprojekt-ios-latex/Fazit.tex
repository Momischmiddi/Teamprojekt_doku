\chapter{Fazit}
\label{cha:faiz}

Unserer Meinung nach wurde das Projekt erfolgreich umgesetzt. Es ist uns gelungen, den Helikopter mit zwei Kameras im Raum zu lokalisieren und den Abstand zum Kamerasystem zu berechnen, was den Hauptanforderungen des Projekts entspricht. Obwohl wir es nicht ganz geschafft haben, ein zweites Stereo-Kamerasystem mit einzubeziehen, haben wir dabei bereits einige Fortschritte machen und einige Ansätze als nicht zielführend erkennen können. Außerdem war nicht mehr genug Zeit, das System hinsichtlich Genauigkeit und Performance zu optimieren. Eine der größten Herausforderungen des Projekts war, dass die Kamera-Kalibrierung, der Grundstein des Projekts, nie endgültig verifiziert und somit als Fehlerquelle ausgeschlossen werden konnte. Trotzdem konnten wir kontinuierlich Fortschritte machen, was die Frustration über die immer wiederkehrenden Rückschläge schnell vergessen machte. Dieses Teamprojekt bot eine hervorragende Gelegenheit, die in den Kursen “Computer Vision” und “Maschinelles Lernen” vorgestellten Techniken praktisch anzuwenden. 
Wie für ein Softwareprojekt üblich, gilt auch für dieses, dass es immer weiter verbessert und erweitert werden kann. Eine Möglichkeit das Projekt zu verbessern, wäre eine genauere Abstandsmessung zu erreichen. Uns ist es leider bisher nicht gelungen, die Ursache für die Ungenauigkeit der Abstandsmessung herauszufinden. Eine mögliche Erweiterung des Projekts könnte die Bestimmung der Ausrichtung im Raum des Helikopters zu bestimmen.