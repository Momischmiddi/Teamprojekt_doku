\chapter{Technologien}
\label{cha:technologien}

\section{Software}
\label{sec:software}

\subsection{OpenCV}
\label{sec:opencv}

OpenCV ist eine Open-Source-Bibliothek, die "uber Algorithmen f"ur maschinelles Sehen und Bildverarbeitung verf"ugt \cite{Ocv}.

\begin{figure}[H]
	\includegraphics[scale=1.0]{bilder/opencv}
	\caption[OpenCV]{OpenCV}
	\cite{Ocv}
	\small Quelle: \url{https://de.wikipedia.org/wiki/OpenCV#/media/Datei:OpenCV_Logo_with_text.png}
\end{figure}

\subsection{scikit-learn}
\label{sec:scikitlearn}

Scikit-learn ist eine freie plattformunabh"angige Python-Bibliothek, die f"ur das maschinelle Lernen konzipiert ist. Die Software ist unter BSD lizenzensiert \cite{ScL}.
Von dieser Bibliothek wird lediglich die Implementierung des k-Means-Algorithmus verwendet.

\begin{figure}[H]
	\includegraphics[scale=0.2]{bilder/scikit}
	\caption[scikit]{scikit}
	\small Quelle: \url{https://upload.wikimedia.org/wikipedia/commons/0/05/Scikit_learn_logo_small.svg}
\end{figure}

\subsection{Open3D}
\label{sec:open3d}

Open3D ist eine Open-Source Bibliothek, die diverse Algorithmen f"ur das Verarbeiten von 3D-Daten bereitstellt.

\subsection{PyCapture}
\label{sec:pycapture}

Mittels PyCapture werden die Kameras angesteuert. Diese Bibliothek liefert 15 Bilder pro Sekunde.

\section{Hardware}
\label{sec:hardware}

\subsection{Kamera}
\label{sec:kamera}
