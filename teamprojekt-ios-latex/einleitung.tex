\chapter{Einleitung}
\label{cha:einleitung}

\section{Aufgabenstellung und Zielsetzung}
\label {sec:aufgabenstellungzielsetzung}

Im Rahmen dieses Teamprojekts stand die Entwicklung eines Mehrbildkamerasystems zur r"aumlichen Detektion eines Modellhubschraubers. Dies beinhaltet sowohl das Erkennen des Helikopters, als auch die Abstandsmessung von diesem.\newline
Dies sollte mit Hilfe Bilderverarbeitungs- und Machine Learning-Techniken, sowie der Verwendung von zwei oder mehr Kameras umgesetzt werden.\newline
Die Lernziele umfassten das Erlernen des Umgangs mit Kameras f"ur die industrielle Bildverarbeitung, ein Verst"andnis f"ur die Grundlagen industrieller Signalverarbeitung zu schaffen. Zudem sollten grundlegende KI-Verfahren erlernt werden.\newline

\section{Motivation}
\label {sec:motivation}

\setlength\epigraphwidth{15cm}
\setlength\epigraphrule{0pt}

\epigraph{\textit{\glqq Computer vision, or the ability of artificially intelligent systems to see like humans, has been a subject of increasing interest and rigorous research for decades now.\grqq{}}}{--- \textup{}Naveen Joshi\cite{NJ}\\}

Das maschinelle Sehen gewinnt in den letzten Jahren immer mehr an Popularit"at. Sei es in der Forschung oder z.B. in der Spieleentwicklung mittels augmented reality.\newline
Durch die steigende Relevanz in der Praxis wurde auch unser Interesse f"ur dieses Themengebiet geweckt. Es ist spannend zu verstehen, wie komplex die Dinge, die f"ur uns Menschen selbstverst"andlich erscheinen, eigentlich sind. Ist es nur das ermitteln der Tiefe eines Objekts im Raum.\newline
Ein weiterer Anreiz f"ur das Projekt waren die verschiedenen angewandten Technologien. Wir alle interessieren und sehr f"ur das Programmieren. Viel Erfahrung in der Programmiersprache Python hatte aber anfangs keines der Teammitglieder. Somit war das Erlernen dieser Sprache eine weitere Motivation.\newline
Auch die zum Gro"steil verwendete Bibliothek OpenCV hat das Interesse an das Projekt geweckt.

\section{Aufbau}
\label {sec:aufbau}
Die Ausarbeitung des Teamprojekts besteht aus drei Teilen.\newline
Anfangs wird kurz auf die angewandten Technologien eingegangen.
Anschlie"send wird die eigentliche Umsetzung und das Vorgehen erl"autert. Zuletzt wird auf die aufgetretenen Probleme eingegangen und ein Fazit gezogen.
