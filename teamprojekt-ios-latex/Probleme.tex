\chapter{Probleme}
\label{cha:probleme}

Trotz der erfolgreichen Lokalisierung unseres Helikopters, hatten wir einige Schwierigkeiten, die auch nicht bis zur Beendigung des Projektes gelöst werden konnten. Das Hauptproblem, mit dem wir nicht gerechnet hatten, war die Kalibrierung und die vielen Folgeprobleme, die sich daraus entwickelten. Nach dem Arbeitspaket Kalibrierung, in dem wir uns eingelesen, selbst kalibriert haben und unsere Methode mit den Methoden von OpenCV verglichen haben, sind wir davon ausgegangen, das Thema sei abgeschlossen. Aber eine plausible Validierung aus den Daten war für uns nicht möglich. Die erhaltenen Werte konnten nicht eingeordnet werden, ob diese gut oder schlecht sind. Zwar versuchten wir dies immer wieder mit neuen Messungen zu bestätigen, aber ein konsistenter Beleg, dass unsere Kalibrierung exakt ist, war nicht möglich. So dachten wir oft, wir hätten dieses Thema erfolgreich beendet, aber bei Fehlschlägen in weiterführenden Themen, wie großen Unstimmigkeiten in den Resultaten, konnten durch eine exaktere Kalibrierung bessere Ergebnisse erreicht werden. Die Ergebnisse schwankten auch sehr stark, je nach Lichteinstrahlung im Labor. So war es oft am besten abends Aufnahmen zu machen, da kein direktes Licht auf unseren Versuchsaufbau fiel. Trotz  abhängen des Fensters war es nicht möglich, dieses Schwanken zu eliminieren. Dies führte auch bei Erfolgen schnell zu einer Demotivation weiterzumachen, da eine andere Fehlerquelle nicht auszuschließen war. Es konnte auch kein Arbeitspaket wirklich beendet werden, da es nicht m"oglich war, dieses als erfolgreich abgeschlossen zu deklarieren.

\noindent Wir haben mit der Kalibrierung von zwei Kameras zueinander gestartet, was viel Zeit in Anspruch genommen hat. Als wir unseren Aufbau um ein weiteres Kamerasystem erweitert haben, war es aufgrund von Einschränkungen durch die verwendete Hardware nicht mehr möglich Bilder aufzunehmen. Dieses Problem konnte durch eine Reduzierung der Bildgröße behoben werden. Als Folge dessen musste jedoch das bereits kalibrierte erste System neu kalibriert werden.

\noindent Des Weiteren war es uns oft nicht möglich aus der OpenCV Dokumentation die richtigen Schlüsse zu ziehen, da sie lückenhaft und teilweise veraltet war. Daraus resultierten Fehler, welche laut der Dokumentation nicht passieren sollten.\newline

\noindent F"ur das Ansteuern der Kameras wurde die Bibliothek ''PyCapture2'' verwendet. Diese hatte aber keine vernünftige Dokumentation, was das Einbinden der Bibliothek deutlich erschwert hat.\newline

\noindent Die f"ur die Entwicklung verwendete Entwicklungsumgebung Spyder hat sehr begrenzte Debug-M"oglichkeiten. So ist es zum Beispiel nicht m"oglich, sich im Debug-Modus gro"se Matrizen anzuzeigen, was die Auswertung von Matrizen sehr verkompliziert hat.
