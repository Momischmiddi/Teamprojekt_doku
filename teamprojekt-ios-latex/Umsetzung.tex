\chapter{Bildverarbeitung und Umsetzung}
\label{cha:verarbeitungumsetzung}

\section{Kalibrierung}
\label{sec:kalibrierung}

F"ur eine Messung, bei der der Fehler minimiert werden soll, ist das Kalibrieren der Kameras unumg"anglich. Durch die Linse einer Kamera entsteht eine tonnenf"ormige Verzeichnung. Diese Fehler sind meist so klein, dass sie vom menschlichen Auge nicht erfasst werden k"onnen \cite{VZ} \cite{VZ1}. Durch die Kalibrierung der Kamera k"onnen diese kompensiert werden.\newline

\begin{figure}[H]
	\includegraphics[scale=0.45]{bilder/verzeichnung}
	\caption[Verzeichnung]{Verzeichnung}
	\small Quelle: \url{http://www.fotokurs-bremen.de/wp-content/uploads/2016/11/Objektiv-Verzeichnung.jpg}
\end{figure}

\noindent Durch die Kamerakalibrierung werden folgende Parameter bestimmt:

\begin{description}
	\item[Intrinsische Parameter]
	Bezeichnen die Abbildung von 3D-Punkten im Kamerakoordinatensystem auf den 2D-Sensor der Kamera. Es sind Informationen der Kamera selbst, die unabh"angig davon sind, wo sich die Kamera befindet und wie diese ausgerichtet ist \cite{Intr}.
	
	\item[Extrinsische Parameter]
	Die r"aumliche Lage und Orientierung der Kamera zu einem Referenzkoordinatensystem, d.h. die Rotation und Translation \cite{cal} \cite{extr}.
\end{description}

\noindent Da es ich bei dem System um ein Stereokamera-System handelt, ist die Kalibrierung von diesem etwas komplizierter.\newline
Zuerst m"ussen die Kameras gesondert kalibriert werden. Dies wird mit der Funktion \textit{calibrateCamera} von OpenCV durchgef"uhrt. F"ur die Kalibrierung wird ein Schachbrett-Muster verwendet. Wichtig ist, dass bei der Kalibrierung beide Kameras dasselbe Bild verwenden. F"ur die Erkennung des Schachbretts wird die OpenCV-Funktion \textit{findChessboardCorners} verwendet. Diese liefert die Objekt- und Bild-Punkte der Aufnahme. Bei den Objekt-Punkten handelt es sich um die 3D-Punkte des Bildes, bei den Bild-Punkten um die 2D-Punkte \cite{OcvD}.
\noindent F"ur eine m"oglichst genaue Kalibrierung werden 50 Bilder verwendet. Anhand dieser wird jede Kamera mittels \textit{calibrateCamera} kalibriert.\newline
Diese Funktion liefert die intrinsisches Parameter in Form von einer 3x3 Kamera-Matrix

\begin{figure}[H]
	\includegraphics[scale=0.75]{bilder/matrix}
\end{figure}

\noindent Wobei $f_{x}$ und $f_{y}$ die Brennweite in Pixeln und $c_{x}$ $c_{y}$ ein Hauptpunkt, der normalerweise in der Bildmitte liegt, ist.

\noindent Die Ergebnisse dieses Vorgangs werden auf dem Computer gespeichert, sodass dieser nicht wiederholt werden muss.\newline
\noindent Anschlie"send wird das Ergebnis der Kalibrierungen an die OpenCV-Funktion \textit{stereoCalibrate} "ubergeben.


\begin{figure}[H]
	\includegraphics[scale=0.4]{bilder/calibration}
	\caption[Kalibrierung]{Kalibrierung}
\end{figure}

\section{Stereo-Kalibrierung}
\label{sec:stereokalibrierun}

\subsection{Fehler "Uberpr"ufung}
\label{sec:fehlertest}